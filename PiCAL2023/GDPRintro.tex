The GDPR establishes a unified framework for data protection across the EU and it is currently stretched over 11 chapters consisting 99 articles.%. Across chapters, it mentions several restrictions for personal data handling. 
However, in this paper, we mainly focus on a few articles that have the potential to be designed within our calculus. We discuss such articles in this section from two perspectives: the data controllers(i.e., organizations that process personal data) and the data subjects( i.e., users whose personal data is being collected and processed).

\emph{The Perspective of Data Subjects:} The GDPR primarily aims to enhance transparency and communication, granting data subjects greater control over their data. Art. [12-23] outlines rights that service providers must enable for data subjects while processing their personal data. Art. 15 (Right of Access) emphasizes that users play a vital role in the processing of personal data. Users are entitled to be informed about specific details regarding the handling of their personal data, including 1) the purposes of processing, 2) recipients of their data, 3) storage duration, 4) access to any personal data concerning them, the ability to obtain a copy, rectify inaccuracies, and object to processing.
As a result, these metadata details should be dynamically accessible within the system at all times. For instance, if a user denies processing for certain purposes,  data access within the system must adhere to these restrictions throughout the data life cycle. Prominently, under Article 17 (Right to be forgotten) for any reason, if a user requests data erasure or if specific consent is no longer valid, the data controller is obligated to promptly delete all data, including backups, snapshots, and replicas, without undue delay.

\emph{The Data Controller Perspective:} 
Art. [24-43], specifies the measures to be taken by controllers and processors when handling user's personal data. Art. 24 (Responsibility of the Controller) states that the controller is solely responsible for having technical measures in place to handle the collected data by the regulation. Art. 25 (Data Protection by Design and Default) specifies that all information systems should be designed, configured, and processed with data protection as a fundamental goal. 
Art. 6.1 (Conditions for Consent)  specifies that the data subject's consent to handle personal data for one or more purposes is essential for lawful processing. Art. 7 specifies % various conditions for consent from data subjects. 
that consent needs to be informed and freely given before processing personal data. The terms and conditions presented to receive consent from data subjects should be clear, intelligible, and understandable and users can modify their %consent . The GDPR also formulates that data subjects should be able to modify their 
consent at any time, and data controllers should facilitate this choice.
% If we rephrase this into a technical solution, any processing or collection of personal data should be attached to the individual consent from the users. Upon withdrawal, the personal data should no longer be used for any processing.
 Art. 5 Sec.~1(b) (Purpose Limitation) imposes restrictions on systems for collecting a vast amount of data for ambiguous and broadly classified purposes. It states that data controllers are compelled to use personal data only for specific, well-defined purposes and cannot process further for alternative purposes. Additionally, Sec.~1(e) of Art. 5 mentions that personal data shall only be stored if necessary and imposes storage limitations based on purposes.


