Notions such as privacy by design (PbD)~\cite{CavoukianC09}, data protection by design and default~\cite{EUA2014}, and legal compliance appear quite often in the literature; despite much ongoing work in this direction, the research community agrees about the need of practical guidance on such notions~\cite{utz__2019, kutylowski_gdpr_2020 }. Schneider indicates that privacy principles, such as PbD, go through various levels of abstraction from their conceptual models during design time until software implementation and do not entirely guarantee privacy~\cite{schneider_is_2018}. Hence, there is a need for privacy-aware languages that help capture software design models with data privacy constructs that facilitate the check of privacy principles.
The purpose based processing data is vital to privacy and formal approaches such as 
privacy-aware role-based access control models consider purposes, roles, and obligations to specify the access control policies and role-based access control (RBAC) models enriched with purpose-awareness~\cite{yang_purpose-based_2007,byun_purpose_2005,masoumzadeh_purbac_2008 } for enterprise data handling have been previously well-explored; these authorization models fail to capture the current regulatory requirements and consider only organizational interests and not user preferences.
%
In existing language-based approaches, a programming language with privacy principles is checked either statically or at runtime. Researchers have previously explored information flow analysis to make  programs comply with  privacy policies, many of these
approaches use static techniques~\cite{TokasOR22, sen_bootstrapping_2014, myers_protecting_2000}, which will not be enough to fully capture %automate 
GDPR compliance, since elements, such as consent change at runtime. 
Some runtime checking work, as in~\cite{TokasO20, KaramiBJ22}, the former considers a subset of GDPR elements that we address and the latter has built in consent management operations but are not considering process calculi.
%
There are well-explored works using pi-calculus in the state-of-the-art reasoning security and privacy properties. $\pi$-calculus serves as both a modeling language for systems and a tool for validating various properties. For instance, applied pi calculus, as discussed in \cite{abadi2017applied}, facilitates the specification and automatic analysis of security protocols Type system have been employed with $ \pi$ calculus in number of papers to reason about access control and policy authorisations however the exploration of the GDPR concepts is not prevalent in these works.. Closest to our work is the Privacy Calculus which proposes a modelling language incorporating the notions like purpose, consent and privacy policies. Nevertheless, in contrast to our approach, Privacy Calculus relies on static techniques to model consent, since consent is dynamic given on the fly static approaches will not be enough to reason about compliance.  They also do not consider retention and deletion of data. In the direction of a general study of GDPR and legal compliance,
Ranise and Siswantoro~\cite{ranise_automated_2017} propose an approach for privacy-aware automated legal compliance checking by using tools for policy analysis on efficient SMT solvers. Unfortunately, this was proposed before GDPR and did not consider crucial elements like consent, actions, entities, and other GDPR requirements. Piras~{et al.}~\cite{piras_defend_2019} propose the design of an architecture abiding by GDPR requirements. However, this approach lacks implementation details on compliance and does not address data interoperability at lower system operations, unlike our approach, where enforcement of data subjects' preferences and monitoring inconsistency with consent is handled on a data level, guaranteeing a concrete notion of compliance.
%Further away from the techniques shown in this paper, there is a line of research that explores compliance checking through blockchain technology; Vargas~\cite{vargas_blockchain-based_nodate} proposes a blockchain-based automated tool for compliance that considers consent operations and the purpose of processing. Similarly, Truong~{et al.}~\cite{truong_gdpr-compliant_2020} propose a Blockchain-based personal data management compliance with the GDPR platform, guaranteeing GDPR requirements like transparency and accountability. However, blockchain's immutability, i.e, ledgers can never be erased, clashes with GDPR's "right to be forgotten", and even if personal data is stored off-chain, any record on the chain will result in violation of GDPR in the platform. \\
%Privacy by Design (PbD)~\cite{CavoukianC09} and data protection by design and default~\cite{EUA2014}, encourage a multidisciplinary approach to identify potential privacy risks and develop effective mitigation strategies. It promotes transparency, accountability, and user-centricity, ensuring that individuals' privacy rights are respected throughout the lifecycle of a product or service. 

%\emph{Formal Methods and privacy : }