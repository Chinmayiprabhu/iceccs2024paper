Software systems often collect and analyze user data to tailor experiences to individual preferences. Extensive interaction of users with software applications has resulted in frequent and broader dissemination of information beyond ways users can comprehend. While personalized services can enhance user satisfaction, it often raises concerns about privacy and data protection. To control this rampant flow of information online and offline data protection regulations such as GDPR came into practice. The GDPR applies to all organizations that handle personal data about EU residents, regardless of the location, and several conditions have to be satisfied to be GDPR-compliant. However, extracting legal requirements and mapping them into software functionality is complex and error-prone. 
For instance, one of the core provisions of GDPR is to ensure that the processing of personal data is according to user's valid will or, in GDPR terms, also referred to as user's consent, this consent as per the GDPR  must be freely given, specific, informed, and unambiguous, but in practice; one might say that this is seldom the case; often, consent is uninformed and presented in unclear forms, which leads users to accept the conditions of the provider without a choice. On the other hand, failure to provide proof of valid consent by the organization may result in noncompliance with GDPR. This higher restriction on consent and data processing requirements has raised a need for the organization to revisit the design and development of its software system to guarantee GDPR compliance.
 In this current work, 
 %we focus on the consent management framework. Since consent sets the lawful basis for data processing, 
we formally model a distributed system with the notion of consent and the other core provisions of GDPR.  We choose pi-calculus as the starting point, which provides a formal and expressive framework to study the dynamic behaviour of distributed %and concurrent 
systems. %We extend the pi-calculus with the consent and storage system, which is available globally at the system level. 
\cp{revise the contributions bit more here}
Intuitively, we associate processes with entities, the data flow within the scope of the entity process must comply with the concerned user's consent. We capture the consent in the form of user policies, which capture the notion of purposes, entities, actions, and respective retention time %allowed on data
enforcing purpose based processing of data. We monitor personal data flow within various entities and check against the consent available at the state. To achieve this, we provide operational semantics in the form of user interaction rules and system rules. 

